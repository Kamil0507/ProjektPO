\chapter{Podsumowanie}
\label{chap:podsumowanie}

Celem pracy było zaprojektowanie i zaimplementowanie w pełni funkcjonalnego, desktopowego Systemu Zarządzania dla jednostki Ochotniczej Straży Pożarnej. Głównym założeniem było stworzenie narzędzia, które usprawni ewidencję kluczowych zasobów i działań, zastępując tradycyjne metody cyfrowym, scentralizowanym rozwiązaniem. Zaimplementowany system stanowi podstawę, którą można w przyszłości rozbudowywać o nowe funkcjonalności. Poniżej przedstawiono kilka potencjalnych kierunków dalszego rozwoju:

\begin{itemize}
    \item \textbf{Moduł finansowy i magazynowy:} Rozszerzenie systemu o możliwość zarządzania budżetem, ewidencjonowania wydatków oraz śledzenia stanu magazynowego sprzętu i umundurowania.
    \item \textbf{Generowanie raportów i statystyk:} Implementacja narzędzi do automatycznego generowania raportów w formacie PDF (np. rocznych sprawozdań z działalności, raportów z interwencji) oraz tworzenia statystyk, które mogłyby wspomóc analizę działań jednostki.
    \item \textbf{Wersja webowa lub mobilna:} Przeniesienie aplikacji na platformę webową lub stworzenie dedykowanej aplikacji mobilnej, co zwiększyłoby dostępność systemu i pozwoliłoby na wprowadzanie danych bezpośrednio w terenie.
    \item \textbf{Kalendarz i harmonogram:} Dodanie modułu kalendarza do planowania szkoleń, zbiórek, przeglądów technicznych sprzętu oraz innych ważnych wydarzeń.
    \item \textbf{System powiadomień:} Wprowadzenie automatycznych powiadomień, np. o zbliżającym się terminie ważności badań lekarskich strażaków lub przeglądów technicznych pojazdów.
\end{itemize}

Realizacja powyższych propozycji mogłaby uczynić system jeszcze bardziej kompleksowym i użytecznym narzędziem, wspierającym OSP w ich codziennej służbie.