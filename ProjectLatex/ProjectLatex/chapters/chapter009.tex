\chapter{Aspekty Techniczne i Zarządzanie Projektem}
\label{chap:aspekty_techniczne}

W tym rozdziale omówiono system kontroli wersji wykorzystany w projekcie oraz przedstawiono wyzwania napotkane podczas implementacji wraz z ich rozwiązaniami.

\section{System Kontroli Wersji i Repozytorium}
\label{sec:git_repo}

Jako platforma do hostingu zdalnego repozytorium kodu został wybrany serwis \textbf{GitHub}. Pełni on nie tylko rolę kopii zapasowej, ale również stanowi publiczne portfolio projektu. Całość kodu źródłowego aplikacji jest dostępna publicznie pod adresem:
\begin{center}
    \url{https://github.com/Kamil0507/ProjektPO}
\end{center}

\section{Problemy Napotkane Podczas Realizacji}
\label{sec:problemy}

Podczas implementacji aplikacji napotkano kilka wyzwań technicznych, których rozwiązanie miało kluczowy wpływ na stabilność i funkcjonalność systemu.

\subsection{Zapewnienie Spójności Danych przy Zapisie Złożonym}
\label{ssec:problem_transakcje}

\paragraph{Problem} Jednym z pierwszych wyzwań było dodawanie nowego strażaka do systemu. Operacja ta wymagała wykonania dwóch oddzielnych zapytań \texttt{INSERT}: jednego do tabeli \texttt{strazacy} i drugiego do tabeli \texttt{uzytkownicy}. Istniało ryzyko, że po pomyślnym dodaniu rekordu do pierwszej tabeli, zapis do drugiej mógłby się nie udaść (np. z powodu błędu połączenia). Pozostawiłoby to bazę danych w niespójnym stanie — istniałby strażak bez powiązanego z nim konta użytkownika.

\paragraph{Rozwiązanie} Aby zagwarantować atomowość tej operacji, zaimplementowano mechanizm \textbf{transakcji bazodanowych}. Przed wykonaniem zapytań, tryb automatycznego zatwierdzania (\texttt{AutoCommit}) jest wyłączany. Oba zapytania \texttt{INSERT} są wykonywane w ramach jednej transakcji. Jeśli oba zakończą się sukcesem, zmiany są trwale zapisywane w bazie za pomocą komendy \texttt{commit()}. W przypadku wystąpienia błędu na dowolnym etapie, wszystkie dotychczasowe zmiany są wycofywane za pomocą \texttt{rollback()}, przywracając bazę do stanu sprzed rozpoczęcia operacji.

\subsection{Agregacja Danych z Wielu Tabel w Jednym Widoku}
\label{ssec:problem_agregacja}

\paragraph{Problem} Wyświetlenie listy interwencji w czytelny sposób wymagało zebrania danych z wielu tabel. Pojedyncza interwencja jest powiązana z wieloma strażakami i wieloma pojazdami poprzez tabele łączące (\texttt{interwencja\_uczestnicy} i \texttt{interwencja\_pojazdy}). Wyzwaniem było przedstawienie listy nazwisk uczestników oraz oznaczeń pojazdów w pojedynczych komórkach głównej tabeli interwencji.

\paragraph{Rozwiązanie} Problem ten rozwiązano na poziomie zapytania SQL, wykorzystując funkcję agregującą \texttt{GROUP\_CONCAT} dostępną w MySQL. Zapytanie zostało skonstruowane w taki sposób, aby grupowało wyniki po identyfikatorze interwencji, a następnie łączyło nazwiska strażaków oraz oznaczenia pojazdów w pojedyncze ciągi znaków, oddzielone przecinkami. Dzięki temu cała logika agregacji danych odbywa się po stronie bazy danych, a aplikacja otrzymuje gotowe do wyświetlenia, sformatowane informacje.

\subsection{Obsługa Edycji Dat w Tabeli}
\label{ssec:problem_daty}

\paragraph{Problem} W module edycji danych strażaka umożliwiono bezpośrednią modyfikację dat (np. ważności badań) w komórkach komponentu \texttt{JTable}. Powstał problem z obsługą formatu danych — po zakończeniu edycji, wartość w komórce mogła być obiektem typu \texttt{String} lub \texttt{java.util.Date}, podczas gdy sterownik JDBC wymagał obiektu typu \texttt{java.sql.Date} do poprawnego zapisu w bazie. Bezpośrednia konwersja groziła błędem \texttt{IllegalArgumentException}.

\paragraph{Rozwiązanie} W klasie \texttt{StrazacyEdytuj} zaimplementowano dedykowaną, prywatną metodę pomocniczą \texttt{convertToSqlDate(Object dateObj)}. Metoda ta w bezpieczny sposób konwertuje otrzymany obiekt na format \texttt{java.sql.Date}. Sprawdza ona typ obiektu wejściowego i obsługuje różne przypadki: jeśli obiekt jest już w poprawnym formacie, jest zwracany bez zmian; jeśli jest to \texttt{java.util.Date}, jest konwertowany; jeśli jest to \texttt{String}, jest parsowany zgodnie z oczekiwanym formatem "yyyy-MM-dd". Takie podejście znacząco zwiększyło odporność aplikacji na błędy związane z wprowadzaniem danych przez użytkownika.


\chapter{Harmonogram Realizacji Projektu}
\label{chap:harmonogram}
Prace rozpoczęto od faza analizy i planowania, która zajęła pierwszy dzień. Drugiego dnia zrealizowano projekt bazy danych. Najdłuższy, trwający od trzeciego do piątego dnia, był etap implementacji kluczowych modułów systemu. Szósty dzień w całości poświęcono na testowanie funkcjonalności i wprowadzanie niezbędnych poprawek. Ostatni, siódmy dzień, przeznaczono na przygotowanie kompletnej dokumentacji technicznej i użytkowej projektu (Rys. \ref{fig:harmonogram_gantta}).

\begin{figure}[H]
    \centering
    \includegraphics[width=\textwidth]{figures/harmonogram_gantta.png}
    \caption{Diagram Gantta przedstawiający tygodniowy harmonogram prac nad projektem.}
    \label{fig:harmonogram_gantta}
\end{figure}