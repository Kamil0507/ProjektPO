\section{Moduł Zarządzania Pojazdami}
\label{chap:pojazdy}

Ostatnim z kluczowych modułów aplikacji jest system ewidencji pojazdów. Umożliwia on katalogowanie oraz zarządzanie flotą pojazdów dostępnych w jednostce. Dostęp do tego modułu, podobnie jak do pozostałych, jest realizowany z poziomu menu głównego, gdzie znajduje się przycisk ,,Pojazdy''. Jego kliknięcie powoduje zamknięcie okna menu i otwarcie interfejsu zarządzania pojazdami.

\section*{Widok Główny Modułu}
\label{sec:pojazdy_widok}

Główny widok modułu (Rys. \ref{fig:pojazdy_lista}) przedstawia listę wszystkich pojazdów wprowadzonych do systemu. Dane prezentowane są w tabeli, która zawiera dwie kluczowe kolumny:
\begin{itemize}
    \item \textbf{Oznaczenie:} Taktyczne lub zwyczajowe oznaczenie pojazdu (np. GBA 3/16 MAN).
    \item \textbf{Numer Rejestracyjny:} Oficjalny numer rejestracyjny pojazdu.
\end{itemize}
Interfejs ten stanowi centralny punkt do przeglądania floty oraz inicjowania operacji dodawania, edycji lub usuwania pojazdów za pomocą przycisków umieszczonych na dole okna.

\begin{figure}[H]
	\centering
	\includegraphics[width=\textwidth]{figures/Pojazdy.png}
	\caption{Główny widok modułu zarządzania pojazdami.}
	\label{fig:pojazdy_lista}
\end{figure}

\section*{Dodawanie Nowego Pojazdu}
\label{sec:pojazdy_dodaj}

Dodawanie nowego pojazdu do ewidencji odbywa się za pomocą prostego formularza (Rys. \ref{fig:pojazdy_dodaj_form}). Formularz zawiera dwa pola tekstowe: ,,Oznaczenie'' i ,,Numer rejestracyjny''. Po ich uzupełnieniu, naciśnięcie przycisku ,,Dodaj'' powoduje zapisanie nowego rekordu w bazie danych. Proces ten, podobnie jak w przypadku dodawania strażaka, jest poprzedzony walidacją sprawdzającą, czy pola nie są puste.

\begin{figure}[H]
	\centering
	\includegraphics[width=0.7\textwidth]{figures/PojazdDodaj.png}
	\caption{Formularz dodawania nowego pojazdu.}
	\label{fig:pojazdy_dodaj_form}
\end{figure}

\section*{Edycja Danych Pojazdu}
\label{sec:pojazdy_edycja}

Modyfikacja istniejących danych o pojeździe jest możliwa w dedykowanym oknie edycji (Rys. \ref{fig:pojazdy_edycja_widok}). Interfejs ten prezentuje listę pojazdów w tabeli, w której można bezpośrednio edytować dane w komórkach. W celu jednoznacznej identyfikacji modyfikowanego rekordu, w widoku tym widoczna jest dodatkowa kolumna ,,ID''. Po wprowadzeniu zmian, użytkownik zatwierdza je przyciskiem ,,Zapisz''.

\begin{figure}[H]
    \centering
    \includegraphics[width=\textwidth]{figures/PojazdyEdytuj.png}
    \caption{Widok edycji danych pojazdu.}
    \label{fig:pojazdy_edycja_widok}
\end{figure}

\section*{Usuwanie Pojazdu}
\label{sec:pojazdy_usuniecie}

System pozwala również na usuwanie pojazdów z ewidencji. W tym celu użytkownik musi przejść do okna usuwania (Rys. \ref{fig:pojazdy_usuniecie_widok}), zaznaczyć wiersz odpowiadający pojazdowi przeznaczonemu do usunięcia, a następnie kliknąć przycisk ,,Usuń''. Zgodnie z przyjętym w aplikacji standardem, operacja ta jest chroniona oknem dialogowym z prośbą o potwierdzenie, aby uniknąć omyłkowego skasowania danych. Usunięcie może być zablokowane, jeśli dany pojazd jest przypisany do istniejącej interwencji, co chroni integralność danych w systemie.

\begin{figure}[H]
    \centering
    \includegraphics[width=\textwidth]{figures/PojazdyUsun.png}
    \caption{Widok usuwania pojazdu z ewidencji.}
    \label{fig:pojazdy_usuniecie_widok}
\end{figure}