\section{Moduł Zarządzania Interwencjami}
\label{chap:interwencje}

Moduł zarządzania interwencjami jest kolejnym kluczowym elementem systemu, umożliwiającym ewidencjonowanie i zarządzanie akcjami ratowniczo-gaśniczymi. Dostęp do modułu jest możliwy z poziomu menu głównego aplikacji, poprzez dedykowany przycisk ,,Interwencje''. Struktura modułu, podobnie jak w przypadku zarządzania strażakami, opiera się na operacjach CRUD (Create, Read, Update, Delete).

\section*{Widok Główny Modułu}
\label{sec:interwencje_widok}

Główny widok modułu (Rys. \ref{fig:interwencje_lista}) prezentuje w formie tabeli listę wszystkich zarejestrowanych interwencji. Umożliwia to szybki przegląd kluczowych informacji o zdarzeniach.


Tabela zawiera następujące kolumny:
\begin{itemize}
    \item \textbf{Data:} Dokładna data i czas rozpoczęcia interwencji.
    \item \textbf{Rodzaj:} Kategoria zdarzenia, np. ,,Miejscowe zagrożenie'' lub ,,Pożar''.
    \item \textbf{Miejsce:} Adres lub lokalizacja, w której miała miejsce interwencja.
    \item \textbf{Opis Działań:} Zwięzły opis podjętych czynności ratowniczych.
    \item \textbf{Pojazdy:} Lista pojazdów zadysponowanych do zdarzenia.
    \item \textbf{Uczestnicy (nazwiska):} Lista strażaków biorących udział w akcji.
\end{itemize}
Na dole okna znajdują się przyciski nawigacyjne: ,,Dodaj'', ,,Edytuj'', ,,Usuń'' oraz ,,Cofnij'', które pozwalają na wykonanie odpowiednich operacji lub powrót do menu głównego.

\begin{figure}[H]
	\centering
	\includegraphics[width=\textwidth]{figures/Interwencje.png}
	\caption{Główny widok modułu zarządzania interwencjami.}
	\label{fig:interwencje_lista}
\end{figure}


\section*{Dodawanie Nowej Interwencji}
\label{sec:interwencje_dodaj}

Po wybraniu opcji ,,Dodaj'', użytkownik jest kierowany do formularza (Rys. \ref{fig:interwencje_dodaj_form}), który umożliwia wprowadzenie danych nowego zdarzenia. Formularz zawiera pola tekstowe odpowiadające kolumnom z widoku głównego. Zwraca uwagę wyraźna informacja dla użytkownika o wymaganym formacie daty i godziny: \texttt{YYYY-MM-DD HH-MM-SS}. Logika działania formularza, analogicznie do modułu strażaków, obejmuje walidację wprowadzonych danych (sprawdzenie, czy kluczowe pola nie są puste) oraz wykonanie zapytania \texttt{INSERT} do bazy danych po naciśnięciu przycisku ,,Dodaj''.

\begin{figure}[H]
	\centering
	\includegraphics[width=0.8\textwidth]{figures/InterwencjeDodaj.png}
	\caption{Formularz dodawania nowej interwencji.}
	\label{fig:interwencje_dodaj_form}
\end{figure}

\section*{Edycja Interwencji}
\label{sec:interwencje_edycja}

Funkcjonalność edycji pozwala na modyfikację istniejących wpisów. Interfejs (Rys. \ref{fig:interwencje_edycja_widok}) prezentuje dane w tabeli, która pozwala na bezpośrednią zmianę wartości w komórkach. W widoku tym pojawia się dodatkowa kolumna ,,ID'', która jednoznacznie identyfikuje rekord w bazie danych. Po dokonaniu pożądanych zmian, użytkownik zatwierdza je przyciskiem ,,Zapisz''. Działanie to inicjuje wykonanie serii zapytań \texttt{UPDATE} w bazie danych w celu uaktualnienia zmodyfikowanych rekordów.

\begin{figure}[H]
    \centering
    \includegraphics[width=\textwidth]{figures/InterwencjeEdytuj.png}
    \caption{Widok edycji interwencji.}
    \label{fig:interwencje_edycja_widok}
\end{figure}

\section*{Usuwanie Interwencji}
\label{sec:interwencje_usuniecie}

System umożliwia również trwałe usuwanie wpisów o interwencjach. W dedykowanym do tego widoku (Rys. \ref{fig:interwencje_usuniecie_widok}) użytkownik musi najpierw zaznaczyć wiersz w tabeli, który chce usunąć, a następnie kliknąć przycisk ,,Usuń''. Analogicznie do pozostałych modułów, operacja ta jest zabezpieczona oknem dialogowym z prośbą o potwierdzenie, aby zapobiec przypadkowemu usunięciu danych. Po potwierdzeniu, wykonywane jest zapytanie \texttt{DELETE} na rekordzie o odpowiednim identyfikatorze (ID).

\begin{figure}[H]
    \centering
    \includegraphics[width=\textwidth]{figures/InterwencjeUsun.png}
    \caption{Widok usuwania interwencji.}
    \label{fig:interwencje_usuniecie_widok}
\end{figure}

